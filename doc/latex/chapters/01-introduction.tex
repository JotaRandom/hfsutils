\section{Overview}

The HFS Utilities (hfsutils) project provides a comprehensive set of tools for working with Apple's HFS (Hierarchical File System) and HFS+ (HFS Plus) filesystems on Unix-like operating systems including Linux, BSD, and macOS.

\subsection{Purpose and Goals}

HFS and HFS+ filesystems are commonly used on:
\begin{itemize}
    \item Classic Mac OS systems (HFS)
    \item Mac OS X and macOS systems prior to APFS (HFS+)
    \item iPod devices
    \item External drives formatted for Mac compatibility
    \item Disk images and backups from Apple systems
\end{itemize}

This toolset enables Unix systems to:
\begin{itemize}
    \item \textbf{Create} HFS and HFS+ filesystems with \texttt{mkfs.hfs} and \texttt{mkfs.hfs+}
    \item \textbf{Check and repair} filesystem integrity with \texttt{fsck.hfs} and \texttt{fsck.hfs+}
    \item \textbf{Mount} HFS and HFS+ volumes with \texttt{mount.hfs} and \texttt{mount.hfs+} (requires kernel support)
    \item \textbf{Manipulate files} on HFS volumes without mounting using \texttt{hfsutil} commands (useful on systems without HFS kernel drivers)
\end{itemize}

\subsection{Supported Systems}

The utilities work on any POSIX-compliant system with appropriate kernel support:

\begin{table}[h]
\centering
\begin{tabular}{lll}
\toprule
\textbf{System} & \textbf{HFS Support} & \textbf{HFS+ Support} \\
\midrule
Linux & Kernel module \texttt{hfs} & Kernel module \texttt{hfsplus} \\
FreeBSD & Native support & Native support \\
macOS & Native support & Native support \\
OpenBSD & Via FUSE & Via FUSE \\
NetBSD & Via FUSE & Via FUSE \\
\bottomrule
\end{tabular}
\caption{Platform Support Matrix}
\end{table}

\textbf{Note}: On systems without kernel HFS support, the \texttt{hfsutil} commands can still be used to access HFS filesystem contents.

\subsection{Key Features}

\subsubsection{Full Specification Compliance}
All utilities strictly adhere to:
\begin{itemize}
    \item Inside Macintosh: Files (HFS specification)
    \item Apple Technical Note TN1150 (HFS+ specification)
    \item Unix/POSIX filesystem utility standards
\end{itemize}

\subsubsection{Zero-Tolerance Validation}
The test suite implements a "zero-tolerance" policy:
\begin{itemize}
    \item ANY deviation from specification = test failure
    \item ALL filesystems must be 100\% correct
    \item Complete validation before and after fsck operations
\end{itemize}

\subsubsection{Journaling Support}
HFS+ journaling is supported with appropriate warnings:
\begin{itemize}
    \item Journal creation with \texttt{mkfs.hfs+ -j}
    \item Journal validation and replay in \texttt{fsck.hfs+}
    \item Linux kernel compatibility warnings (journaling not supported in Linux HFS+ driver)
\end{itemize}

\subsubsection{Date Limit Awareness}
The utilities handle filesystem date limits correctly:
\begin{itemize}
    \item HFS: Maximum date February 6, 2028 (Y2K28 problem)
    \item HFS+: Maximum date February 6, 2040 (Y2K40 problem)
    \item Automatic date correction to safe values
\end{itemize}

\section{Installation}

\subsection{Building from Source}

\subsubsection{Prerequisites}
\begin{lstlisting}[style=bashstyle]
# Debian/Ubuntu
sudo apt-get install build-essential

# Fedora/RHEL
sudo dnf install gcc make

# macOS
xcode-select --install

# BSD
# Compiler included by default
\end{lstlisting}

\subsubsection{Compilation}
\begin{lstlisting}[style=bashstyle]
# Clone repository
git clone https://github.com/JotaRandom/hfsutils.git
cd hfsutils

# Build all utilities
make

# Build specific sets
make set-hfs        # mkfs.hfs, fsck.hfs, mount.hfs
make set-hfsplus    # mkfs.hfs+, fsck.hfs+, mount.hfs+

# Build with hfsutil
make all
\end{lstlisting}

\subsubsection{Installation Options}

\textbf{Linux systems} (with kernel HFS/HFS+ drivers):
\begin{lstlisting}[style=bashstyle]
sudo make install-linux PREFIX=/usr
\end{lstlisting}

\textbf{Complete installation} (filesystem utilities + hfsutil):
\begin{lstlisting}[style=bashstyle]
sudo make install-complete PREFIX=/usr/local
\end{lstlisting}

\textbf{Individual utilities}:
\begin{lstlisting}[style=bashstyle]
sudo make install-mkfs.hfs+ PREFIX=/usr
sudo make install-fsck.hfs PREFIX=/usr
sudo make install-mount.hfs+ PREFIX=/usr
\end{lstlisting}

\subsection{Verifying Installation}

After installation, verify the utilities are available:
\begin{lstlisting}[style=bashstyle]
mkfs.hfs --version
mkfs.hfs+ --version
fsck.hfs --version
fsck.hfs+ --version
mount.hfs --help
mount.hfs+ --help
hfsutil --version   # If installed
\end{lstlisting}

Check manpages:
\begin{lstlisting}[style=bashstyle]
man mkfs.hfs
man mkfs.hfs+
man fsck.hfs+
man mount.hfs+
\end{lstlisting}

\section{Quick Start}

\subsection{Creating a Filesystem}

Create an HFS filesystem:
\begin{lstlisting}[style=bashstyle]
# Create 10MB image file
dd if=/dev/zero of=test.img bs=1M count=10

# Format as HFS
mkfs.hfs -L "MyDisk" test.img
\end{lstlisting}

Create an HFS+ filesystem:
\begin{lstlisting}[style=bashstyle]
# Create 50MB image file
dd if=/dev/zero of=test.img bs=1M count=50

# Format as HFS+
mkfs.hfs+ -L "MyDisk" test.img

# Format with journaling (Linux warning will appear)
mkfs.hfs+ -j -L "MyDisk" test.img
\end{lstlisting}

\subsection{Checking a Filesystem}

Verify filesystem integrity:
\begin{lstlisting}[style=bashstyle]
# Check HFS filesystem (read-only)
fsck.hfs -n test.img

# Check and repair HFS+ filesystem
fsck.hfs+ -y test.img

# Verbose output
fsck.hfs+ -v -n test.img
\end{lstlisting}

\subsection{Mounting a Filesystem}

On systems with kernel HFS/HFS+ support:
\begin{lstlisting}[style=bashstyle]
# Create mount point
sudo mkdir /mnt/hfs

# Mount read-write
sudo mount.hfs+ test.img /mnt/hfs

# Mount read-only
sudo mount.hfs+ -r test.img /mnt/hfs

# Unmount
sudo umount /mnt/hfs
\end{lstlisting}

\subsection{Using hfsutil Commands}

On systems without kernel support or for direct access:
\begin{lstlisting}[style=bashstyle]
# Format volume
hfsutil hformat -L "MyDisk" test.img

# Mount in hfsutil
hfsutil hmount test.img

# List contents
hfsutil hls

# Copy file into volume
hfsutil hcopy myfile.txt :myfile.txt

# Copy file out of volume
hfsutil hcopy :myfile.txt retrieved.txt

# Unmount
hfsutil humount
\end{lstlisting}

\section{Documentation Structure}

This manual is organized as follows:

\begin{description}
    \item[Chapter 2: HFS Specification] Details of the classic HFS filesystem format
    \item[Chapter 3: HFS+ Specification] Details of the HFS+ filesystem format with journaling
    \item[Chapter 4: mkfs Utilities] Filesystem creation tools
    \item[Chapter 5: fsck Utilities] Filesystem checking and repair tools
    \item[Chapter 6: mount Utilities] Filesystem mounting tools
    \item[Chapter 7: hfsutil Commands] File manipulation without mounting
    \item[Chapter 8: Implementation Details] Internal architecture and algorithms
    \item[Chapter 9: Testing and Validation] Test suite and quality assurance
    \item[Chapter 10: Appendix] Structure definitions, error codes, glossary
\end{description}
