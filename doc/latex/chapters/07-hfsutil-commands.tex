\chapter{hfsutil Commands}

This chapter documents the hfsutil suite of commands for interactive manipulation of HFS/HFS+ volumes without mounting.

\section{Overview}

hfsutil provides a set of Unix-like commands for working with HFS and HFS+ volumes directly, similar to mtools for FAT filesystems. All commands operate on unmounted volumes.

\subsection{Common Features}
\begin{itemize}
    \item Work on HFS and HFS+ volumes
    \item No mounting required
    \item Unix-style command interface
    \item Support for both block devices and image files
    \item MacRoman to UTF-8 conversion
\end{itemize}

\section{Volume Management Commands}

\subsection{hformat - Format HFS Volume}

\textbf{Synopsis}:
\begin{verbatim}
hformat [-l label] device
\end{verbatim}

\textbf{Description}: Creates a new HFS Classic volume (equivalent to mkfs.hfs).

\textbf{Options}:
\begin{itemize}
    \item -l label: Set volume label (1-27 characters)
    \item device: Block device or file to format
\end{itemize}

\textbf{Example}:
\begin{verbatim}
hformat -l "BackupDisk" /dev/sdb1
hformat -l "Test" disk.hfs
\end{verbatim}

\subsection{hmount - Mount HFS Volume (hfsutil context)}

\textbf{Synopsis}:
\begin{verbatim}
hmount device [partition]
\end{verbatim}

\textbf{Description}: Register HFS volume for use with other hfsutil commands (does NOT mount to OS).

\textbf{Options}:
\begin{itemize}
    \item device: Block device or image file
    \item partition: Optional partition number (for APM)
\end{itemize}

\textbf{Example}:
\begin{verbatim}
hmount /dev/sdb1
hmount disk.hfs
hmount /dev/sdc 2  # Partition 2
\end{verbatim}

\subsection{humount - Unmount HFS Volume (hfsutil context)}

\textbf{Synopsis}:
\begin{verbatim}
humount [device]
\end{verbatim}

\textbf{Description}: Unregister HFS volume from hfsutil.

\textbf{Example}:
\begin{verbatim}
humount
humount /dev/sdb1
\end{verbatim}

\subsection{hvol - Display Volume Information}

\textbf{Synopsis}:
\begin{verbatim}
hvol [device]
\end{verbatim}

\textbf{Description}: Show volume information (name, creation date, free space, etc.).

\textbf{Example}:
\begin{verbatim}
hvol
# Output:
# Volume name: MyDisk
# Volume created: Mon Jan 15 12:00:00 2024
# Total blocks: 10240
# Free blocks: 5120
\end{verbatim}

\section{File Operations Commands}

\subsection{hls - List Directory}

\textbf{Synopsis}:
\begin{verbatim}
hls [-1abcdefgilmnrstux] [path ...]
\end{verbatim}

\textbf{Description}: List files and directories (like Unix ls).

\textbf{Common Options}:
\begin{itemize}
    \item -l: Long format (permissions, size, date)
    \item -a: Show all files (including invisible)
    \item -i: Show file IDs (CNIDs)
    \item -d: List directory itself, not contents
    \item -R: Recursive listing
    \item -1: One file per line
\end{itemize}

\textbf{Examples}:
\begin{verbatim}
hls                    # List current directory
hls -l                 # Long format
hls -la /              # All files in root, long format
hls -R /System         # Recursive list
\end{verbatim}

\subsection{hcopy - Copy Files}

\textbf{Synopsis}:
\begin{verbatim}
hcopy [-m|-b|-t|-r] source-path target-path
hcopy [-m|-b|-t|-r] source-path [...] target-directory
\end{verbatim}

\textbf{Description}: Copy files to/from HFS volume.

\textbf{Fork Options}:
\begin{itemize}
    \item -m: MacBinary format (default, includes both forks)
    \item -b: Copy both data and resource forks
    \item -t: Text mode (data fork only, line ending conversion)
    \item -r: Raw mode (data fork only, no conversion)
\end{itemize}

\textbf{Examples}:
\begin{verbatim}
# Copy from HFS to Unix:
hcopy :file.txt ./file.txt
hcopy -r :document.doc ./document.doc

# Copy to HFS:
hcopy ./readme.txt :README.TXT
hcopy ./file1 ./file2 :Folder/

# MacBinary format (preserves resource fork):
hcopy -m :app.bin ./app.bin
\end{verbatim}

\subsection{hmkdir - Create Directory}

\textbf{Synopsis}:
\begin{verbatim}
hmkdir path [...]
\end{verbatim}

\textbf{Description}: Create new directories on HFS volume.

\textbf{Example}:
\begin{verbatim}
hmkdir :Documents
hmkdir :Files :Backup :Archives
\end{verbatim}

\subsection{hdel - Delete Files/Directories}

\textbf{Synopsis}:
\begin{verbatim}
hdel path [...]
\end{verbatim}

\textbf{Description}: Delete files and directories (directories must be empty).

\textbf{Example}:
\begin{verbatim}
hdel :oldfile.txt
hdel :TempFolder
hdel :file1.doc :file2.doc
\end{verbatim}

\subsection{hrename - Rename Files}

\textbf{Synopsis}:
\begin{verbatim}
hrename src-path dest-path
\end{verbatim}

\textbf{Description}: Rename or move files/directories within volume.

\textbf{Example}:
\begin{verbatim}
hrename :oldname.txt :newname.txt
hrename :File.doc :Archive/File.doc
\end{verbatim}

\section{Navigation Commands}

\subsection{hcd - Change Directory}

\textbf{Synopsis}:
\begin{verbatim}
hcd [path]
\end{verbatim}

\textbf{Description}: Change current directory in HFS volume.

\textbf{Example}:
\begin{verbatim}
hcd :Documents
hcd ..
hcd /
\end{verbatim}

\subsection{hpwd - Print Working Directory}

\textbf{Synopsis}:
\begin{verbatim}
hpwd
\end{verbatim}

\textbf{Description}: Display current directory path.

\textbf{Example}:
\begin{verbatim}
hp

wd
# Output: :Documents:Projects
\end{verbatim}

\section{Attribute Commands}

\subsection{hattrib - Show/Modify Attributes}

\textbf{Synopsis}:
\begin{verbatim}
hattrib [-t TYPE] [-c CREA] [-i|+i] path [...]
\end{verbatim}

\textbf{Description}: Display or modify file/folder attributes (type, creator, invisible flag).

\textbf{Options}:
\begin{itemize}
    \item -t TYPE: Set file type (4 characters)
    \item -c CREA: Set creator (4 characters)
    \item -i: Make invisible
    \item +i: Make visible
\end{itemize}

\textbf{Examples}:
\begin{verbatim}
hattrib :file.txt              # Show attributes
hattrib -t TEXT -c EDIT :file.txt  # Set type/creator
hattrib -i :HiddenFile         # Make invisible
hattrib +i :NowVisible          # Make visible
\end{verbatim}

\section{Path Syntax}

\textbf{HFS path format}:
\begin{itemize}
    \item : prefix indicates HFS volume path
    \item : separator between folders (not /)
    \item Example: :Folder:Subfolder:file.txt
    \item / is NOT a path separator in HFS
    \item Root: : or :/
\end{itemize}

\textbf{Examples}:
\begin{verbatim}
:                    # Root directory
:Documents           # Documents folder in root
:Docs:file.txt       # file.txt in Docs folder
..                   # Parent directory
\end{verbatim}

\section{Character Encoding}

\textbf{HFS uses MacRoman encoding}:
\begin{itemize}
    \item hfsutil converts MacRoman $\leftrightarrow$ UTF-8 automatically
    \item Some characters may not convert perfectly
    \item Colon (:) forbidden in filenames (path separator)
\end{itemize}

\section{Workflow Example}

\textbf{Complete workflow for accessing HFS volume}:
\begin{verbatim}
# 1. Mount volume to hfsutil
hmount /dev/sdb1

# 2. Navigate and explore
hpwd
hls -la
hcd :Documents

# 3. Copy files
hcopy :file.txt ./backup/
hcopy ./newfile.doc :

# 4. Create directory and organize
hmkdir :Archive
hrename :oldfile.txt :Archive/oldfile.txt

# 5. Cleanup
hdel :tempfile.tmp

# 6. Unmount
humount
\end{verbatim}

\section{Testing}

See \texttt{test/test\_hfsutils.sh} for comprehensive tests:
\begin{itemize}
    \item Format, mount, unmount sequence
    \item File creation and deletion
    \item Directory operations
    \item Copy to/from volume
    \item Attribute manipulation
    \item Path handling
\end{itemize}
