\chapter{Mount Utilities}

This chapter documents mount.hfs and mount.hfs+ utilities for mounting HFS and HFS+ filesystems following Unix/BSD/Linux standards.

\section{mount.hfs and mount.hfs+ - Filesystem Mounting}

\subsection{Synopsis}

\begin{verbatim}
mount.hfs [-o options] [-r] [-w] [-v] device mountpoint
mount.hfs+ [-o options] [-r] [-w] [-v] device mountpoint
mount -t hfs [-o options] device mountpoint
mount -t hfsplus [-o options] device mountpoint
\end{verbatim}

\subsection{Description}

Mount HFS or HFS+ filesystems at specified mountpoints. These utilities follow Unix mount(8) conventions and integrate with /etc/fstab. Automatically validate filesystem before mounting and handle both read-only and read-write modes.

\subsection{Standard Options}

\begin{longtable}{lp{10cm}}
\toprule
\textbf{Option} & \textbf{Description} \\
\midrule
\endhead
-o options & Mount options (comma-separated, see below) \\
-r & Mount read-only (shorthand for -o ro) \\
-w & Mount read-write (shorthand for -o rw) \\
-v & Verbose output \\
device & Block device to mount (e.g., /dev/sdb1) \\
mountpoint & Directory where filesystem will be mounted \\
\bottomrule
\caption{mount.hfs/mount.hfs+ Options}
\end{longtable}

\subsection{Mount Options (-o)}

\begin{longtable}{lp{10cm}}
\toprule
\textbf{Option} & \textbf{Description} \\
\midrule
\endhead
ro & Read-only mount (default for journaled on Linux) \\
rw & Read-write mount \\
noexec & Do not allow execution of binaries \\
nosuid & Do not honor setuid/setgid bits \\
nodev & Do not interpret device files \\
sync & Synchronous I/O (slow but safe) \\
async & Asynchronous I/O (fast but risky on crash) \\
uid=N & Set owner UID for all files \\
gid=N & Set group GID for all files \\
umask=N & Set umask for file permissions \\
force & Force mount even if filesystem appears dirty \\
\bottomrule
\caption{Common Mount Options}
\end{longtable}

\subsection{Mount Process}

\subsubsection{Pre-Mount Validation}
\begin{enumerate}
    \item Verify device exists and is accessible
    \item Check if device is already mounted
    \item Read filesystem signature (offset 1024)
    \item Validate signature (0x4244 for HFS, 0x482B/0x4858 for HFS+)
    \item Check unmounted bit in attributes
    \item If dirty: warn and suggest fsck
    \item Verify mountpoint exists and is empty directory
\end{enumerate}

\subsubsection{Kernel Module Check (Linux)}
\begin{verbatim}
# Check if kernel module loaded
lsmod | grep hfs
# Load if needed
modprobe hfs      # For HFS classic
modprobe hfsplus  # For HFS+
\end{verbatim}

\subsubsection{Mount Execution}
\begin{enumerate}
    \item Call kernel mount() syscall
    \item Pass filesystem type ("hfs" or "hfsplus")
    \item Pass mount options
    \item Kernel driver validates and mounts
    \item Update /etc/mtab (on success)
    \item Return exit code
\end{enumerate}

\subsection{Journaling Considerations}

\textbf{CRITICAL for Linux users}:

\begin{itemize}
    \item Linux HFS+ driver does NOT support journaling
    \item Journaled volumes MUST be mounted read-only on Linux
    \item Automatic detection: if journaled bit set → force ro
    \item Warning displayed: "Volume is journaled, mounting read-only"
\end{itemize}

\textbf{Detection algorithm}:
\begin{verbatim}
if (attributes & 0x00002000) {
    fprintf(stderr, "WARNING: Journaled volume\n");
    fprintf(stderr, "Linux does not support HFS+ journaling\n");
    fprintf(stderr, "Mounting read-only\n");
    force_readonly = 1;
}
\end{verbatim}

\subsection{Exit Codes}

\begin{longtable}{lp{10cm}}
\toprule
\textbf{Code} & \textbf{Meaning} \\
\midrule
\endhead
0 & Success, filesystem mounted \\
1 & Incorrect invocation or permissions \\
2 & System error (out of memory, etc.) \\
4 & Internal mount bug \\
8 & User interrupt \\
16 & Problems writing or locking /etc/mtab \\
32 & Mount failure \\
64 & Some mount succeeded \\
\bottomrule
\caption{mount Exit Codes (Linux Standard)}
\end{longtable}

\subsection{Examples}

\textbf{Mount HFS classic read-write}:
\begin{verbatim}
mount.hfs /dev/sdb1 /mnt/hfs
mount -t hfs /dev/sdb1 /mnt/hfs
\end{verbatim}

\textbf{Mount HFS+ read-only}:
\begin{verbatim}
mount.hfs+ -r /dev/sdc1 /mnt/backup
mount -t hfsplus -o ro /dev/sdc1 /mnt/backup
\end{verbatim}

\textbf{Mount with specific permissions}:
\begin{verbatim}
mount.hfs+ -o uid=1000,gid=1000,umask=022 /dev/sdd1 /mnt/data
\end{verbatim}

\textbf{Force mount dirty volume (WARNING - DANGEROUS)}:
\begin{verbatim}
mount.hfs+ -o force /dev/sde1 /mnt/recovery
# WARNING: Can cause data corruption!
\end{verbatim}

\subsection{Unmounting}

\textbf{Standard unmount}:
\begin{verbatim}
umount /mnt/hfs
umount /dev/sdb1
\end{verbatim}

\textbf{Force unmount (if busy)}:
\begin{verbatim}
umount -f /mnt/hfs
# Or lazy unmount:
umount -l /mnt/hfs
\end{verbatim}

\textbf{Check what's using mount}:
\begin{verbatim}
lsof | grep /mnt/hfs
fuser -m /mnt/hfs
\end{verbatim}

\subsection{/etc/fstab Integration}

\textbf{HFS entry}:
\begin{verbatim}
/dev/sdb1  /mnt/hfs  hfs  defaults,ro  0  0
\end{verbatim}

\textbf{HFS+ entry}:
\begin{verbatim}
/dev/sdc1  /mnt/backup  hfsplus  ro,noexec,nosuid  0  0
\end{verbatim}

\textbf{HFS+ with UUID}:
\begin{verbatim}
UUID=xxxx-xxxx  /mnt/data  hfsplus  rw,uid=1000  0  2
\end{verbatim}

\textbf{Field meanings}:
\begin{enumerate}
    \item Device or UUID
    \item Mount point
    \item Filesystem type (hfs or hfsplus)
    \item Mount options
    \item Dump frequency (usually 0)
    \item fsck pass number (0=skip, 1=root, 2=other)
\end{enumerate}

\subsection{Common Issues}

\subsubsection{Device is Busy}
\begin{verbatim}
mount: /dev/sdb1 already mounted or /mnt/hfs busy
\end{verbatim}
\textbf{Solutions}:
\begin{itemize}
    \item Check if already mounted: \texttt{mount | grep sdb1}
    \item Find processes using it: \texttt{lsof | grep sdb1}
    \item Kill processes or use \texttt{umount -f}
\end{itemize}

\subsubsection{Wrong Filesystem Type}
\begin{verbatim}
mount: wrong fs type, bad option, bad superblock
\end{verbatim}
\textbf{Solutions}:
\begin{itemize}
    \item Verify signature: \texttt{xxd -s 1024 -l 2 /dev/sdb1}
    \item Use correct type: hfs vs hfsplus
    \item Run fsck first: \texttt{fsck.hfs+ /dev/sdb1}
\end{itemize}

\subsubsection{Permission Denied}
\begin{verbatim}
mount: only root can do that
\end{verbatim}
\textbf{Solution}: Use sudo
\begin{verbatim}
sudo mount.hfs+ /dev/sdb1 /mnt/hfs
\end{verbatim}

\subsubsection{Module Not Found}
\begin{verbatim}
mount: unknown filesystem type 'hfsplus'
\end{verbatim}
\textbf{Solution}: Load kernel module
\begin{verbatim}
sudo modprobe hfsplus
# Or for HFS classic:
sudo modprobe hfs
\end{verbatim}

\subsection{Portability Notes}

\textbf{Linux}: Full support via kernel modules (hfs.ko, hfsplus.ko)
\begin{itemize}
    \item HFS: Full read/write support
    \item HFS+: Read/write support, NO journaling
    \item Always mount journaled volumes read-only
\end{itemize}

\textbf{FreeBSD}: Native HFS read support
\begin{itemize}
    \item Write support via FUSE
    \item Mount with: \texttt{mount -t hfs /dev/da0s1 /mnt}
\end{itemize}

\textbf{macOS}: Native full support
\begin{itemize}
    \item Automatic mounting via diskutil
    \item Full journaling support
    \item Default filesystem (legacy, now APFS)
\end{itemize}

\section{Implementation}

\subsection{Source Code}
\begin{itemize}
    \item \texttt{src/mount/mount\_hfs.c}: HFS mount helper
    \item \texttt{src/mount/mount\_hfsplus.c}: HFS+ mount helper
    \item \texttt{src/mount/mount\_common.h}: Shared definitions
\end{itemize}

\subsection{Critical Functions}

\textbf{validate\_filesystem()}:
\begin{verbatim}
int validate_filesystem(const char *device) {
    int fd = open(device, O_RDONLY);
    uint16_t sig;
    lseek(fd, 1024, SEEK_SET);
    read(fd, &sig, 2);
    sig = be16toh(sig);
    
    if (sig == 0x4244) return FS_HFS;
    if (sig == 0x482B || sig == 0x4858) return FS_HFSPLUS;
    return FS_UNKNOWN;
}
\end{verbatim}

\textbf{check\_journaling()}:
\begin{verbatim}
int check_journaling(int fd) {
    uint32_t attributes;
    lseek(fd, 1024 + 4, SEEK_SET);
    read(fd, &attributes, 4);
    attributes = be32toh(attributes);
    return (attributes & 0x00002000) ? 1 : 0;
}
\end{verbatim}
