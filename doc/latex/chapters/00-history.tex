\section{Overview}

This chapter traces the evolution of Apple's filesystems from the original Macintosh File System (MFS) through HFS, HFS+, and the modern APFS. Understanding this evolution provides context for design decisions and compatibility requirements.

\section{Macintosh File System (MFS)}

\subsection{Introduction}

MFS (Macintosh File System) was the original filesystem used on the Macintosh 128K and 512K (1984-1985). It was designed for floppy disks and small hard drives.

\subsection{Key Characteristics}

\begin{itemize}
    \item \textbf{Introduced}: January 1984 (Macintosh 128K)
    \item \textbf{Maximum volume size}: 20 MB (limited by hardware)
    \item \textbf{Maximum files}: 65,535 (16-bit file numbers)
    \item \textbf{Filename length}: 255 characters (Macintosh Roman)
    \item \textbf{Directory structure}: Flat (no folders/subdirectories)
    \item \textbf{Dual forks}: Data fork and resource fork
    \item \textbf{File types}: 4-character type and creator codes
\end{itemize}

\subsection{Design Philosophy}

MFS was revolutionary for its time:
\begin{itemize}
    \item First consumer filesystem with resource forks
    \item Integrated with Finder for desktop metaphor
    \item Type/creator codes enabled double-click launching
    \item Desktop file tracked file positions and icons
\end{itemize}

\subsection{Limitations}

\begin{itemize}
    \item \textbf{No hierarchy}: All files stored in single directory
    \item \textbf{Finder folders}: Simulated by Finder, not filesystem
    \item \textbf{Small volumes}: Designed for 400K floppies
    \item \textbf{Poor scalability}: Performance degraded with many files
\end{itemize}

\subsection{MFS Structure Summary}

\begin{table}[h]
\centering
\begin{tabular}{ll}
\toprule
\textbf{Component} & \textbf{Description} \\
\midrule
Logical Block 0-1 & Boot blocks (1024 bytes) \\
Logical Block 2 & Master Directory Block (MDB) \\
Following blocks & File directory (single B-tree) \\
... & Allocation bitmap \\
... & File data \\
Last sector & Alternate MDB \\
\bottomrule
\end{tabular}
\caption{MFS Volume Layout}
\end{table}

\textbf{Note}: MFS is not implemented in hfsutils but documented for historical context and understanding HFS evolution.

\section{Hierarchical File System (HFS)}

\subsection{Introduction}

HFS (Hierarchical File System) replaced MFS in Mac OS System 2.1 (1985) and System 3.0 (1986). It introduced true hierarchical directories.

\subsection{Key Improvements over MFS}

\begin{itemize}
    \item \textbf{Hierarchical directories}: True folder structure
    \item \textbf{Larger volumes}: Up to 2 GB (later 2 TB with HFS wrapper)
    \item \textbf{Better performance}: Separate B-trees for catalog and extents
    \item \textbf{Allocation optimization}: Clump sizes reduce fragmentation
\end{itemize}

\subsection{Timeline}

\begin{itemize}
    \item \textbf{1985}: Introduced with Hard Disk 20
    \item \textbf{1986}: Became standard in System 3.0
    \item \textbf{1998}: Superseded by HFS+ in Mac OS 8.1
    \item \textbf{1999}: Last use in Mac OS 9 for booting
    \item \textbf{2000s}: Maintained for compatibility
    \item \textbf{Today}: Supported for legacy media
\end{itemize}

\subsection{HFS Characteristics}

Detailed in Chapter 2 (HFS Specification). Summary:

\begin{table}[h]
\centering
\begin{tabular}{ll}
\toprule
\textbf{Feature} & \textbf{Specification} \\
\midrule
Signature & 0x4244 ('BD') \\
Maximum volume & 2 TB \\
Maximum file & 2 GB \\
Filename length & 31 characters (MacRoman) \\
Allocation blocks & 16-bit addressing (65,535 blocks) \\
B-tree node size & 512 bytes (fixed) \\
Date range & 1904-2028 (Y2K28 limit) \\
Case sensitivity & Case-insensitive, case-preserving \\
\bottomrule
\end{tabular}
\caption{HFS Characteristics}
\end{table}

\section{HFS Plus (HFS+)}

\subsection{Introduction}

HFS+ (HFS Plus, "Mac OS Extended") was introduced in Mac OS 8.1 (1998) to address HFS limitations for modern computing.

\subsection{Major Enhancements}

\begin{itemize}
    \item \textbf{Unicode filenames}: UTF-16, up to 255 characters
    \item \textbf{32-bit addressing}: Support for very large volumes
    \item \textbf{Smaller allocation blocks}: Better space efficiency
    \item \textbf{Journaling}: Optional crash recovery (Mac OS X 10.2.2+)
    \item \textbf{Hard links}: Unix-style hard links
    \item \textbf{Symbolic links}: Unix-style symbolic links
    \item \textbf{Extended attributes}: Arbitrary metadata
    \item \textbf{HFSX variant}: Case-sensitive option
\end{itemize}

\subsection{Timeline}

\begin{itemize}
    \item \textbf{1998}: Introduced in Mac OS 8.1
    \item \textbf{1999}: Became default in Mac OS 8.6
    \item \textbf{2001}: Mac OS X adoption
    \item \textbf{2002}: Journaling added (Mac OS X 10.2.2)
    \item \textbf{2005}: Case-sensitive HFSX variant
    \item \textbf{2017}: Superseded by APFS (macOS 10.13)
    \item \textbf{Today}: Still widely used, especially on spinning drives
\end{itemize}

\subsection{HFS+ Variants}

\subsubsection{Standard HFS+}
\begin{itemize}
    \item Signature: 0x482B ('H+')
    \item Version: 4
    \item Case-insensitive filenames
    \item Most compatible
\end{itemize}

\subsubsection{HFSX (HFS Extended)}
\begin{itemize}
    \item Signature: 0x4858 ('HX')
    \item Version: 5
    \item Case-sensitive filenames
    \item Used for Unix-like behavior
\end{itemize}

\subsubsection{Journaled HFS+}
\begin{itemize}
    \item Attributes bit 13 (0x2000) set
    \item Journal info block pointer in Volume Header
    \item Circular journal buffer
   \item \textbf{Not supported by Linux kernel}
\end{itemize}

\subsection{HFS+ Characteristics}

Detailed in Chapter 3 (HFS+ Specification). Summary:

\begin{table}[h]
\centering
\begin{tabular}{ll}
\toprule
\textbf{Feature} & \textbf{Specification} \\
\midrule
Signature & 0x482B ('H+') or 0x4858 ('HX') \\
Maximum volume & 8 EB theoretical \\
Maximum file & 8 EB theoretical \\
Filename length & 255 UTF-16 characters \\
Allocation blocks & 32-bit addressing \\
B-tree node size & 4096 bytes (typical) \\
Date range & 1904-2040 (Y2K40 limit) \\
Case sensitivity & Optional (HFSX) \\
Journaling & Optional \\
\bottomrule
\end{tabular}
\caption{HFS+ Characteristics}
\end{table}

\section{Apple File System (APFS)}

\subsection{Introduction}

APFS (Apple File System) is Apple's modern filesystem, introduced in macOS 10.13 High Sierra (2017). It replaces HFS+ as the default for SSDs and flash storage.

\subsection{Design Goals}

\begin{itemize}
    \item \textbf{Flash optimized}: Minimize write amplification
    \item \textbf{Space sharing}: Multiple volumes share space pool
    \item \textbf{Snapshots}: Instant, space-efficient snapshots
    \item \textbf{Cloning}: Copy-on-write file/directory clones
    \item \textbf{Strong encryption}: Native full-disk and per-file encryption
    \item \textbf{Crash protection}: Copy-on-write metadata
\end{itemize}

\subsection{Key Features}

\begin{itemize}
    \item 64-bit inode numbers
    \item Nanosecond timestamp precision
    \item Space sharing between volumes
    \item Native encryption (FileVault)
    \item Fast directory sizing
    \item Atomic safe-save operations
    \item Clones and snapshots
\end{itemize}

\subsection{APFS vs HFS+}

\begin{table}[h]
\centering
\begin{tabular}{lll}
\toprule
\textbf{Feature} & \textbf{HFS+} & \textbf{APFS} \\
\midrule
Introduced & 1998 & 2017 \\
Optimized for & HDDs & SSDs/Flash \\
Snapshots & No & Yes \\
Cloning & No & Yes (instant) \\
Encryption & Per-volume & Per-file + volume \\
Space sharing & No & Yes \\
Timestamps & Second precision & Nanosecond precision \\
Max files & ~4 billion & ~9 quintillion \\
Journaling & Optional & Always (COW) \\
\bottomrule
\end{tabular}
\caption{HFS+ vs APFS Comparison}
\end{table}

\subsection{APFS Adoption}

\begin{itemize}
    \item \textbf{macOS 10.13+}: Default for SSDs
    \item \textbf{iOS 10.3+}: Default for all devices
    \item \textbf{watchOS 4+}: Default
    \item \textbf{tvOS 11+}: Default
    \item \textbf{HDDs}: HFS+ still recommended (as of macOS 13)
\end{itemize}

\subsection{Why HFS+ Still Matters}

\begin{itemize}
    \item \textbf{Legacy systems}: Pre-2017 Macs
    \item \textbf{HDDs}: APFS not optimized for spinning drives
    \item \textbf{External drives}: Better compatibility
    \item \textbf{Recovery partitions}: Some use HFS+
    \item \textbf{Boot Camp}: Windows compatibility
    \item \textbf{Linux support}: Better HFS+ kernel support
\end{itemize}

\section{Filesystem Evolution Summary}

\begin{table}[h]
\centering
\small
\begin{tabular}{lllll}
\toprule
\textbf{FS} & \textbf{Year} & \textbf{Max Volume} & \textbf{Max File} & \textbf{Key Innovation} \\
\midrule
MFS & 1984 & 20 MB & N/A & Resource forks \\
HFS & 1985 & 2 TB & 2 GB & Hierarchy \\
HFS+ & 1998 & 8 EB & 8 EB & Unicode, large files \\
APFS & 2017 & 8 EB & 8 EB & Flash-optimized, snapshots \\
\bottomrule
\end{tabular}
\caption{Apple Filesystem Evolution}
\end{table}

\section{Scope of This Document}

This manual focuses on:
\begin{itemize}
    \item \textbf{HFS (Classic)}: Complete implementation (Chapter 2)
    \item \textbf{HFS+}: Complete implementation (Chapter 3)
    \item \textbf{MFS}: Historical context only
    \item \textbf{APFS}: Overview for migration context
\end{itemize}

All specifications are documented at the bit level to enable complete reimplementation without external references.

\section{Historical Oddities and Notes}

\subsection{The "BD" Signature Mystery}

HFS uses signature 0x4244 ('BD' in ASCII). The origin is unclear:
\begin{itemize}
    \item Possibly "Block Device"
    \item Possibly arbitrary choice
    \item Never officially documented by Apple
\end{itemize}

\subsection{The 1904 Epoch}

All Apple filesystems use January 1, 1904 as time zero:
\begin{itemize}
    \item Predates Unix epoch (1970) by 66 years
    \item Reason unknown but consistent across all Apple filesystems
    \item Creates Y2K28 (HFS) and Y2K40 (HFS+) problems
\end{itemize}

\subsection{The Colon as Path Separator}

Classic Mac OS used colon (:) as path separator:
\begin{itemize}
    \item Unix/Windows use / and \textbackslash respectively
    \item Filenames cannot contain colons
    \item Causes issues when sharing files cross-platform
    \item Mac OS X translates : to / for display
\end{itemize}

\subsection{The 31 vs 255 Character Limit}

\begin{itemize}
    \item HFS: 31 characters (historical, tied to original Mac)
    \item HFS+: 255 characters (modern standard)
    \item MFS: Actually 255 characters (more than HFS!)
\end{itemize}

\subsection{Case Sensitivity Confusion}

\begin{itemize}
    \item HFS/HFS+: Case-insensitive by default
    \item HFSX: Case-sensitive variant
    \item Linux ext4: Case-sensitive
    \item Windows NTFS: Case-insensitive but preserving
    \item Source of many cross-platform file issues
\end{itemize}
